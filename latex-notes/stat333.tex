\documentclass[10pt]{article} 

\usepackage{fullpage}
\usepackage{bookmark}
\usepackage{amsmath}
\usepackage{amssymb}
\usepackage[dvipsnames]{xcolor}
\usepackage{hyperref} % for the URL
\usepackage[shortlabels]{enumitem}
\usepackage{mathtools}
\usepackage[most]{tcolorbox}
\usepackage[amsmath,standard,thmmarks]{ntheorem} 
\usepackage{physics}
\usepackage{pst-tree} % for the trees
\usepackage{verbatim} % for comments, for version control
\usepackage{tabu}
\usepackage{tikz}
\usepackage{float}

\lstnewenvironment{python}{
\lstset{frame=tb,
language=Python,
aboveskip=3mm,
belowskip=3mm,
showstringspaces=false,
columns=flexible,
basicstyle={\small\ttfamily},
numbers=none,
numberstyle=\tiny\color{Green},
keywordstyle=\color{Violet},
commentstyle=\color{Gray},
stringstyle=\color{Brown},
breaklines=true,
breakatwhitespace=true,
tabsize=2}
}
{}

\lstnewenvironment{cpp}{
\lstset{
backgroundcolor=\color{white!90!NavyBlue},   % choose the background color; you must add \usepackage{color} or \usepackage{xcolor}; should come as last argument
basicstyle={\scriptsize\ttfamily},        % the size of the fonts that are used for the code
breakatwhitespace=false,         % sets if automatic breaks should only happen at whitespace
breaklines=true,                 % sets automatic line breaking
captionpos=b,                    % sets the caption-position to bottom
commentstyle=\color{Gray},    % comment style
deletekeywords={...},            % if you want to delete keywords from the given language
escapeinside={\%*}{*)},          % if you want to add LaTeX within your code
extendedchars=true,              % lets you use non-ASCII characters; for 8-bits encodings only, does not work with UTF-8
% firstnumber=1000,                % start line enumeration with line 1000
frame=single,	                   % adds a frame around the code
keepspaces=true,                 % keeps spaces in text, useful for keeping indentation of code (possibly needs columns=flexible)
keywordstyle=\color{Cyan},       % keyword style
language=c++,                 % the language of the code
morekeywords={*,...},            % if you want to add more keywords to the set
% numbers=left,                    % where to put the line-numbers; possible values are (none, left, right)
% numbersep=5pt,                   % how far the line-numbers are from the code
% numberstyle=\tiny\color{Green}, % the style that is used for the line-numbers
rulecolor=\color{black},         % if not set, the frame-color may be changed on line-breaks within not-black text (e.g. comments (green here))
showspaces=false,                % show spaces everywhere adding particular underscores; it overrides 'showstringspaces'
showstringspaces=false,          % underline spaces within strings only
showtabs=false,                  % show tabs within strings adding particular underscores
stepnumber=2,                    % the step between two line-numbers. If it's 1, each line will be numbered
stringstyle=\color{GoldenRod},     % string literal style
tabsize=2,	                   % sets default tabsize to 2 spaces
title=\lstname}                   % show the filename of files included with \lstinputlisting; also try caption instead of title
}
{}

% floor, ceiling, set
\DeclarePairedDelimiter{\ceil}{\lceil}{\rceil}
\DeclarePairedDelimiter{\floor}{\lfloor}{\rfloor}
\DeclarePairedDelimiter{\set}{\lbrace}{\rbrace}
\DeclarePairedDelimiter{\iprod}{\langle}{\rangle}

\DeclareMathOperator{\Int}{int}
\DeclareMathOperator{\mean}{mean}

% commonly used sets
\newcommand{\R}{\mathbb{R}}
\newcommand{\N}{\mathbb{N}}
\newcommand{\Q}{\mathbb{Q}}
\renewcommand{\P}{\mathbb{P}}

\newcommand{\sset}{\subseteq}

\theoremstyle{break}
\theorembodyfont{\upshape}

\newtheorem{thm}{Theorem}[subsection]
\tcolorboxenvironment{thm}{
enhanced jigsaw,
colframe=Dandelion,
colback=White!90!Dandelion,
drop fuzzy shadow east,
rightrule=2mm,
sharp corners,
before skip=10pt,after skip=10pt
}

\newtheorem{cor}{Corollary}[thm]
\tcolorboxenvironment{cor}{
boxrule=0pt,
boxsep=0pt,
colback={White!90!RoyalPurple},
enhanced jigsaw,
borderline west={2pt}{0pt}{RoyalPurple},
sharp corners,
before skip=10pt,
after skip=10pt,
breakable
}

\newtheorem{lem}[thm]{Lemma}
\tcolorboxenvironment{lem}{
enhanced jigsaw,
colframe=Red,
colback={White!95!Red},
rightrule=2mm,
sharp corners,
before skip=10pt,after skip=10pt
}

\newtheorem{ex}[thm]{Example}
\tcolorboxenvironment{ex}{% from ntheorem
blanker,left=5mm,
sharp corners,
before skip=10pt,after skip=10pt,
borderline west={2pt}{0pt}{Gray}
}

\newtheorem*{pf}{Proof}
\tcolorboxenvironment{pf}{% from ntheorem
breakable,blanker,left=5mm,
sharp corners,
before skip=10pt,after skip=10pt,
borderline west={2pt}{0pt}{NavyBlue!80!white}
}

\newtheorem{defn}{Definition}[subsection]
\tcolorboxenvironment{defn}{
enhanced jigsaw,
colframe=Cerulean,
colback=White!90!Cerulean,
drop fuzzy shadow east,
rightrule=2mm,
sharp corners,
before skip=10pt,after skip=10pt
}

\newtheorem{prop}[thm]{Proposition}
\tcolorboxenvironment{prop}{
boxrule=0pt,
boxsep=0pt,
colback={White!90!Green},
enhanced jigsaw,
borderline west={2pt}{0pt}{Green},
sharp corners,
before skip=10pt,
after skip=10pt,
breakable
}

\setlength\parindent{0pt}
\setlength{\parskip}{2pt}


\begin{document}
\let\ref\Cref

\title{\bf{STAT333 Applied Probability}}
\date{\today}
\author{Austin Xia}

\maketitle
\newpage
\tableofcontents
\listoffigures
\listoftables
\newpage
\section{Course Information}
    \subsection{Contact}
        \begin{center}
            Instructor: Steve Drekic\\
            Email: sdrekic@uwaterloo.ca
        \end{center}
    \subsection{Grading Scheme}
        \begin{center}
            4 assignments 100\%...\\
        \end{center}
\section{}
    \subsection{Review of Elementary Probablity}
        \begin{defn}[Probability Function]
            For each event A of a sample space S, P(A) is defined as the probability of event A, satisfying 3 conditions:
            \begin{itemize}
                \item $0\leq P(A) \leq 1$
                \item p(S)=1
                \item $P(\cup_{i=1}^nA_i)=\sum^n_{i=1}P(A_i)$ if the sequence of events $A_i$ are mutually exclusive
            \end{itemize}
        \end{defn}

        \begin{defn}[independent]
            X and Y are independent rvs if $f(x,y)=f(x)f(y)$
        \end{defn}
        \begin{defn}[multivariate mgf]
            $\phi_{x,y}(a,b)=E(e^{ax+by})$
        \end{defn}
        \begin{thm}
            if $X_1, X_2...X_n$ are independent rvs where $\phi_{X_i}(t)$ is the mgf of $X_i, i = 1,2,...n$.
            then $T=\sum^n_{i=1}X_i$ has mgf $\phi_x(t)=\prod_{i=1}^n\phi_{X_i}(t)$
        \end{thm}
        \begin{thm}[Strong Law of Large Numbers]
            If $X_1, ..., X_n$ is an iid sequence of rvs, each having mean $\mu < \infty$, then with probability 1, as $n\rightarrow \infty$ 
            $$\bar{X}_n=\frac{X_1+X_2 ... + X_n}{n}\rightarrow \mu$$              
        \end{thm}
    \subsection{Conditional Distributions and Conditional Expectation}
        \begin{thm}[Law of total Expectation]
            For rvs X and Y, 
            $$E[g(X)]=E[E[g(x)|Y]]$$
        \end{thm}
        \begin{thm}
            For rvs X and Y,
            $$Var(X)=E[Var(X|Y)]+Var(E[X|Y])$$
        \end{thm}
    \subsection{Computing Probability by conditioning}
        recall 
        $$E[X]=E[E[X|Y]]= \sum_y E[X|Y=y]p_Y(y)$$
        similarly 
        $$P(A)=\sum_yP(A|Y=y)p_Y(y)$$
        $P(X<Y)$ is good example of application
    \section{Discrete-time Markov Chain}
        \begin{defn}[Stochastic Process]
            $\{X(t),t\in T\}$ is called a stochastic process if $X(t)$ is a rv for any $t\in T$
        \end{defn}
        \begin{defn}[DTMC]
            A stochastic process $\{X_n,n\in N\}$ is said to be a discrete-time Markov chain (DTMC) if:
            \begin{itemize}
                \item $X_n$ is a discrete rv for all $n\in N$
                \item the Markov property hold:
                $$P(X_{n+1}=x_{n+1}|X_n=x_n,X_{n-1}=x_{n-1}...X_0=x_0)=P(X_{n+1}=x_{n+1}|X_n=x_n)$$
            \end{itemize}
        \end{defn}
        \begin{defn}[transition probability Matrix] 
            the transition probability from state i at time n to state j at time n+1 is:
            $$P_{nij}=P(X_{n+1}=j|X_n=i),\ n\in\N$$
            the transition probability matrix is all number of $P^{(n)}_{i,j}$
            \\we can show by linear algebra that:$$P^{(n)}=P^n$$
        \end{defn}
        We will only be dealing with stationary or homogeneous situation we at every step we have the same P
        \begin{defn}[$\alpha_n$]
            The marginal pmf of $X_n$ is $\alpha_n=(\alpha_{n0}...\alpha_{nk})$
            \\where $\alpha_{nk}=P(X_n=k)\forall k\in\N$
        \end{defn}
        \subsection{transiant and recurrance}
            \begin{defn}
                The probability that starting from state i, FIRST visit to j occurs 
                at time n is 
                $$f_{i,j}^{(n)}=P(X_n=j, X_{X-1}\neq j ...x_1 \neq j|X_0=i)$$
            \end{defn}

            note $f_{i,j}^{(1)}=P_{i,j}$
            \begin{defn}[transiant]
                State i is transient if $f_{i,i}<1$ otherwise, state i is recurrent
            \end{defn}
            \begin{defn}[$M_i$]
                $M_i$ is a rv which counts the number of times the DTMC visits state i
            \end{defn}
            $$P(M_i=k|X_0=i)=f_{i,i}^k(1-f_{i,i})$$
            visit k times, then never visit it again. We also notice this is geometric Distribution
            so $$E[M_i|X_0=i]=E[Y-1]=\frac{1}{1-f_{i,i}}-1$$
            so if $f_{i,i}=1$, $E[M_i|X_0=i]=\infty$, the state is recurrent
            we can derive that $$E[M_i|X_0=i]=\sum P_{i,i}^{(n)}$$
            \begin{thm}
                if $i\leftrightarrow j$ and state i is recurrent, than state j is recurrent 
            \end{thm}
            \begin{thm}
                if $i\leftrightarrow j$ and state i is recurrent, than $f_{i,j}=1$
            \end{thm}

            \begin{thm}
                inconclusion, if sates are within the same communication class, then\\
                \begin{itemize}
                    \item the states communicate with each other 
                    \item these states all have the same period
                    \item these sates are either recurrent or all transiant
                \end{itemize}
            \end{thm}
            \begin{thm}
                A finite state DTMC has at least one recurrent state
            \end{thm}
            \begin{thm}
                for any i and transient state j of DTMC, $$lim_{n\rightarrow \infty}P_{i,j}^{(n)}=0$$
            \end{thm}
            \begin{defn}
                $N_i=min\{n\in Z:X_n=i\}$
                mean recurrent time $$m_i=E[N_i|X_0=i]=\sum nf_{i,i}^{(n)}$$
            \end{defn}
            \begin{defn}
                Suppose i is recurrent. i is positively recurrent iff $m_i < \infty$, 
                i is nullrecurrent iff $m_i=\infty$
            \end{defn}
            \begin{thm}[Positive and null recurrence]
                \begin{itemize}
                    \item if $i\leftrightarrow j$ and i is positively recurrent then j is also
                    \item in finite state DTMC, there can never be null recurrent states 
                \end{itemize}
            \end{thm}

            \begin{defn}[stationary distribution]
                if $\sum^\infty_{i=0}p_i=1$ and $p_j=\sum_{i=0}^\infty p_iP_{i,j}$\\
                or in matrix form $$p(1,1,1...)=1\ p=pP$$
            \end{defn}
            \begin{thm}
                if irreducible DTMC is positive reccurent iff stationary distribution exisist
                \\Stationary distributions are not nessarily unique
            \end{thm}
            \begin{thm}[Basic Limit thm]
                For an irredicible, recurrent, aperioadic DTMC, $lim_{n\rightarrow \infty}$ exisist and independent of state i satisfying
                $$lim_{n\rightarrow\infty}P_{i,j}^{(n)}=\pi_j=\frac{1}{m_j} \forall i,j\in N$$
                If it is also positive recurrent, then $\pi_j$ is the unique positive solution to 
                $$\pi_j=\sum \pi_i P_{i,j}\ \sum \pi_j=1$$
                
            \end{thm}
        \subsection{Galton Waston Branching Process}
            $Z_i^{(j)}$ be the number of offspringt produced from individual i in jth generation
            $$X_n=\sum_{i=1}^{X_{n-1}}Z_i^{n-1}$$
            since $X_n$ is a random sum, we get 
            $$E[X_n]=\mu E[X_{n-1}]$$
            $$\pi_0=\sum_{j=0}^\infty \pi_0^j\alpha_j$$
            this is summing $x_0$ has 0 to infty offsprings
                
    \section{The Exponential Distribution and the Poisson Process}
        
    $$Y=min\{X_1, ... X_n\} \vdash EXP(\sum_{i=1}^n\lambda_i)$$
    $$P(X_1<X_2...<X_n)=\prod_{i=1}^{n-1}P(X_i=min\{X_i, X_{i+1}...\})$$
    \begin{thm}
        A rv X is memoryless iff $$P(x>y+z|X>y)=P(X>z)\forall y,z\geq 0$$
    \end{thm}

    \begin{thm}
        A rv X is memoryless iff $$P(x>y+z)=P(X>y)*P(X>z)\forall y,z\geq 0$$
    \end{thm}

    Erlang Distribution:
    $$\phi_x(t)=(\frac{\lambda}{\lambda -t})^n$$ which is product of n mgf of $exp(\lambda)$
    \newpage
    \subsection{poisson Process}
    \begin{defn}[counting process]
        A counting process is a stochastic process in which N(t) represents number of event by time t 
    \end{defn}
    \begin{defn}[independent increments]
        A counting process has independent increments if $N(t_1)-N(S_1)$ is independent of 
        $N(t_2)-N(s_2)$ whenever the intervels do not overlap
    \end{defn}
    \begin{defn}[stationary increments]
        A counting process has independent increments if $N(s+t)-N(s)$ dependes only on t 
    \end{defn}
    \begin{defn}[o(h)]
        A function y=f(x) is said to be "o(h)" (of order h) if 
        $$lim_{h\rightarrow0}\frac{f(h)}{h}=0$$
    \end{defn}

    \begin{defn}[Poisson process]
        a counting process is a Poisson process at rate $\lambda$ if 
        \begin{itemize}
            \item the process has independent and stationary increments 
            \item $P(N(h)=1)=\lambda h +o(h)$
            \item $P(N(h)\geq 2)=o(h)$
        \end{itemize}
    \end{defn}

    \begin{thm}
        if $\{N(t), t\geq 0\}$ is a Poisson process at rate $\lambda$, 
        $N(t)\sim  POI(\lambda t)$
    \end{thm}
$$\neg\exists xA(x) \vdash \forall x \neg A(x)$$
    \begin{thm}
        if $\{N(t), t\geq 0\}$ is a Poisson process, then $\{T_i\}$ is a suquence of iid $EXP(\lambda)$
    \end{thm}
\end{document}
